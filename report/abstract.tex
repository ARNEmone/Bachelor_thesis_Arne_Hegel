
% This file contains an abstract of your thesis, with approximaltely 300-500 words

%The abstract has to be given in German \textbf{and} English. It should be between half a page and one page in length. It should cover in a readable and comprehensive style the context of the thesis, the problem setting, the objectives, and the methods developed in this thesis as well as key insights and results.


Supporting the independence of an increasingly aging society is becoming more and more relevant. An electric wheelchair with assistance functions can make this possible in the area of mobility. Especially motorically impaired people can be supported by such a wheelchair. In order to support the wheelchair user in situations where he needs it, this situation must first be identified. One way to accomplish this is to continuously evaluating the wheelchair user's steering behavior. However, this evaluation must take into account the non-holonomic constraints of the wheelchair.

In this way, the problem is two-fold. The first subproblem is to compute a path from intermediate goals to a previously determined goal as short as possible on a given map. This path must be fast, intuitive and navigable. The second subproblem is to compute the distance along the previously computed path in order to measure the evolution of the distance. The distance function must reflect the non-holonomic constraints of the wheelchair. It should then be possible to evaluate the steering behavior with the help of the distances.

In order to realize this, first a graph is generated on the navigable space using a sampling-based algorithm, whose edges can be traversed with a wheelchair. On this graph, an optimal path to a goal previously selected by the driver is then calculated using the A* algorithm. During the drive, the distance to the goal is continuously calculated along the path. The calculated distances can be compared with the previous calculations and thus provide a way to evaluate the driver's steering behavior.

In an experimental evaluation it was shown that with the help of this evaluation method of the steering behavior can be evaluated very well in most cases. However, since the graph consists of randomly selected nodes, misinterpretation of the steering behavior may occur.



