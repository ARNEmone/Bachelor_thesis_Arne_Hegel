
% This file contains the German version of your abstract, with about 300-500 words

%Die Zusammenfassung muss auf Deutsch \textbf{und} auf Englisch geschrieben werden. Die Zusammenfassung sollte zwischen einer halben und einer ganzen Seite lang sein. Sie soll den Kontext der Arbeit, die Problemstellung, die Zielsetzung und die entwickelten Methoden sowie Erkenntnisse bzw.~Ergebnisse �bersichtlich und verst�ndlich beschreiben.

Die Aufrechterhaltung der Selbstst�ndigkeit einer immer �lter werdenden Gesellschaft gewinnt zunehmend an Relevanz. Ein elektrischer Rollstuhl mit Assistenzfunktionen kann dies in gewissen Bereichen erm�glichen. Um den Rollstuhlfahrer in Situationen zu unterst�tzen, in denen er diese ben�tigt, muss diese Situation zun�chst erkannt werden. Eine M�glichkeit, um dies zu bewerkstelligen, ist es das Steuerverhalten des Rollstuhlfahrers fortlaufend zu bewerten. Diese Bewertung muss jedoch die nicht-holonomen Zwangsbeschr�nkungen des Rollstuhls beachten.

Das Problem ist somit zweiteilig. Das erste Teilproblem besteht darin, auf einer gegebenen Karte m�glichst schnell einen Weg aus Zwischenzielen zu einem vorher bestimmten Ziel zu berechnen. Dieser Weg muss dabei schnell, intuitiv und navigierbar sein. Das zweite Teilproblem besteht darin, die Distanz entlang des zuvor berechneten Weges fortlaufen zu berechnen. Die Distanzfunktion muss dabei die nicht-holonomen Zwangsbeschr�nkungen des Rollstuhls widerspiegeln. Mit Hilfe der Distanzen soll daraufhin eine Bewertung des Steuerverhaltens vorgenommen werden k�nnen.

Um dies zu realisieren wird zun�chst mit Hilfe eines stichprobenbasierten Algorithmus ein Graph auf dem navigierbaren Bereich generiert, dessen Kanten mit einem Rollstuhl abgefahren werden k�nnen. Auf diesem Graphen wird daraufhin unter Verwendung des A* Algorithmus ein optimaler Weg zu einem Ziel berechnet, welches der Fahrer zuvor ausgew�hlt hat. W�hrend der Fahrt wird entlang des Weges fortlaufend die Distanz zum Ziel berechnet. Die berechneten Distanzen k�nnen mit den vorherigen Berechnungen verglichen werden und bieten somit eine M�glichkeit das Steuerverhalten des Fahrers zu bewerten.

In einer experimentellen Evaluation wurde gezeigt, dass mit Hilfe dieser Bewertungsmethode des Steuerverhaltens in den meisten F�llen sehr gut bewertet werden kann. Da der Graph jedoch aus zuf�llig ausgew�hlten Knoten besteht, kann es unter Umst�nden zu Fehlinterpretationen des Steuerverhaltens kommen.

